%-------------------------
% Resume in Latex
% Author : Ran Cheng
% Adapted from: Indu dwivedi, Sourabh Bajaj
% License : MIT
%------------------------

\documentclass[letterpaper,10pt]{article}

\usepackage{latexsym}
\usepackage[empty]{fullpage}
\usepackage{titlesec}
\usepackage{marvosym}
\usepackage[usenames,dvipsnames]{color}
\usepackage{verbatim}
\usepackage{enumitem}
\usepackage[pdftex, hidelinks]{hyperref}
\usepackage{fancyhdr}
\usepackage[charter]{mathdesign} % Bitstream Charter
% \usepackage{newpxtext,newpxmath} % Palatino
\usepackage{longtable}
\usepackage{graphicx}
\usepackage{array}
\usepackage{multirow}
\usepackage{xcolor}
\usepackage{bibentry}
\pagestyle{fancy}
\fancyhf{} % clear all header and footer fields
\fancyfoot{}
\renewcommand{\headrulewidth}{0pt}
\renewcommand{\footrulewidth}{0pt}

% bibtex for publication
\bibliographystyle{plain}
\nobibliography{resume.bib}

% Adjust margins
\addtolength{\oddsidemargin}{-0.50in}
\addtolength{\evensidemargin}{-0.50in}
\addtolength{\textwidth}{1in}
\addtolength{\topmargin}{-.5in}
\addtolength{\textheight}{1.0in}

% Define colors
\definecolor{linkblue}{RGB}{111, 153, 222}
\definecolor{titleblue}{RGB}{46, 116, 181}
\urlstyle{same}

\raggedbottom
\raggedright
\setlength{\tabcolsep}{0in}

% Sections formatting
\titleformat{\section}{
  \vspace{-6pt}\scshape\raggedright\large
}{}{0em}{}[\color{black}\titlerule \vspace{-5pt}]

%-------------------------
% Custom commands
\newcommand{\resumeItem}[2]{
  \item\small{
    \textbf{#1}{: #2 \vspace{-2pt}}
  }
}

\newcommand{\resumeItemNoBullet}[2]{
  \item[]\small{
    \hspace{-9pt}\textbf{#1}{: #2 \vspace{-6pt}}
  }
}

\newcommand{\resumeSubheading}[4]{
  \vspace{-1pt}\item[]
  \begin{tabular*}{0.98\textwidth}{l@{\extracolsep{\fill}}r}
      \hspace{-10pt}\textbf{#1} & #2 \\
      \hspace{-10pt}\textit{\small#3} & \textit{\small #4} \\
    \end{tabular*}\vspace{-5pt}
}

\newcommand{\resumeSubheadingone}[2]{
	\vspace{-1pt}\item[]
	\begin{tabular*}{0.98\textwidth}{l@{\extracolsep{\fill}}r}
		\hspace{-10pt}\textbf{#1} & #2 \\
	\end{tabular*}\vspace{-5pt}
}

\newcommand{\resumeSubheadingtwo}[2]{
	\vspace{-1pt}\item[]
	\begin{tabular*}{0.98\textwidth}{l@{\extracolsep{\fill}}r}
		\hspace{-10pt}{#1} & #2 \\
	\end{tabular*}\vspace{-5pt}
}

\newcommand{\resumeSubItem}[2]{\resumeItem{#1}{#2}\vspace{-4pt}}

\renewcommand{\labelitemii}{$\circ$}

\newcommand{\resumeSubHeadingListStart}{\begin{itemize}[leftmargin=*]}
\newcommand{\resumeSubHeadingListEnd}{\end{itemize}}
\newcommand{\resumeItemListStart}{\begin{itemize}}
\newcommand{\resumeItemListEnd}{\end{itemize}\vspace{-5pt}}

% custom commands
\newcommand{\shorterSection}[1]{\vspace{-10pt}\section{#1}}

%-------------------------------------------
%%%%%%  CV STARTS HERE  %%%%%%%%%%%%%%%%%%%%%%%%%%%%


\begin{document}

%----------HEADING-----------------
% you can generate your own qr code here: https://www.the-qrcode-generator.com/
% and convert the svg image you exported to pdf here: https://cloudconvert.com/svg-to-pdf
% then import the graph in the title like this:

\begin{table}[]
\begin{tabular*}{\textwidth}{lc@{\extracolsep{\fill}}r}
\begin{tabular}{l}
\textbf{\huge \textcolor{titleblue}{Shengwu Zhao}} \\
\\
An Integrated Navigation PH.D. Candidate from BIT  
\end{tabular}  &  & \begin{tabular}{@{}rr@{}} \textcolor{titleblue}{\includegraphics[width=0.017\linewidth]{imgs/location(1).pdf}} Beijing Institute of Technology, Beijing, China & \multirow{3}{*}{\includegraphics[width=0.086\linewidth]{imgs/githubpage.png}} \\
\includegraphics[width=0.017\linewidth]{imgs/email(1).pdf}
zhaoshengwu@bit.edu.cn                   &                   \\
\includegraphics[width=0.017\linewidth]{imgs/home(1).pdf} \href{https://s5zhao.github.io/}{s5zhao.github.io}                            &                   \\
\includegraphics[width=0.017\linewidth]{imgs/phone(1).pdf} +8618811693687                                  &                   
\end{tabular}  \\ 
\end{tabular*}
\end{table}

\vspace*{-10mm}


%-----------EDUCATION-----------------
\shorterSection{\textcolor{titleblue}{Education}}
  \resumeSubHeadingListStart
    \resumeSubheading
      {Beijing Institute of Technology, School of Automation}{Beijing, China}     {PH.D. in Control Science and Engineering}{Sep 2020 - Expected Jun 2026}{
    %   \resumeItemNoBullet{Thesis}{Guided Robust Visual Navigation with Deep Learning}
      \resumeItemNoBullet{Major}{Navigation, Guidance and Control}
    }
    \resumeSubheading
      {Beijing Institute of Technology, School of Xuteli}{Beijing, China}     {Bachelor of  Engineering;  Ranking:4/13}{Aug 2016 - Jun 2020}
  \resumeSubHeadingListEnd
  
  %-----------Addtional Experience & Achievements-----------------
\shorterSection{\textcolor{titleblue}{Publications}}
  \resumeSubHeadingListStart
    \resumeSubheadingone
  		{Journal Paper}{}
  \small
  \item{\textbf{Zhao S}, Shi L, Zhang W, Deng Z. Global dynamic path‐planning algorithm in gravity‐aided inertial navigation system[J]. IET Signal Processing, 2021, 15(8): 510-520. (SCI Q3, IF: 1.7) \href{https://ietresearch.onlinelibrary.wiley.com/doi/10.1049/sil2.12056}{\textcolor{linkblue}{[paper]}}}
  \vspace{-5pt}
  \item{\textbf{Zhao S}, Xiao X, Wang Y, Deng Z. An improved particle filter based on gravity measurement feature in gravity-aided inertial navigation system[J].IEEE Sensors Journal, 2022, 23(2): 1423-1435. (SCI Q1, IF: 4.3) \href{https://ieeexplore.ieee.org/document/9979757}{\textcolor{linkblue}{[paper]}}}
  \vspace{-5pt}
  \item{\textbf{Zhao S}, Xiao X, Deng Z, Shi L. Gravity matching algorithm based on correlation filter[J]. IEEE Sensors Journal, 2022, 23(3): 2618-2629. (SCI Q1, IF: 4.3) \href{https://ieeexplore.ieee.org/document/9997486}{\textcolor{linkblue}{[paper]}}}
  \vspace{-5pt}
  \item{\textbf{Zhao S}, Xiao X, Pang X, Wang Y, Deng Z. Gravity Matching Algorithm Based on Backtracking for Small Range Adaptation Area[J]. IEEE Transactions on Instrumentation and Measurement, 2024, 73, 9504413. (SCI Q1, IF: 5.6) \href{https://ieeexplore.ieee.org/document/10381855}{\textcolor{linkblue}{[paper]}}}
  \vspace{-5pt}
  \item{\textbf{Zhao S}, Deng Z, Wang Q, Zhang W, Gong X. Terrain Matching Algorithm Based on Trajectory Reconstruction and Correlation Analysis of Sliding Measurement Sequence[J]. IEEE/ASME Transactions on Mechatronics, 2024. (SCI Q1, IF: 6.1)}
  \vspace{-5pt}
  \item{Wang Y, Deng Z, Zhang P, Wang B, \textbf{Zhao S}. A Gravity-Aided Navigation Matching Algorithm Based on Triangulation[J]. IEEE Sensors Journal, 2024. (SCI Q1, IF: 4.3)}  
	\resumeSubheadingone
	{Conference Paper}{}
	\small
	\item{Wang Y, Deng Z, Zhang W, \textbf{Zhao S}. An improved ICCP gravity matching algorithm based on Mahalanobis distance[C]//2021 40th Chinese Control Conference (CCC). IEEE, 2021: 3503-3508. (EI) \href{https://ieeexplore.ieee.org/document/9550017}{\textcolor{linkblue}{[paper]}}}
	\vspace{-5pt}
	\item{\textbf{Zhao S}, Pang X, Deng Z.  Gravity Matching Algorithm Based on Backtracking[C]// 2022China Inertia Technology Symposium, 2022, Dalian, China.}
	\vspace{-5pt}
	\item{\textbf{Zhao S}, Zhang W, Wang Y, Deng Z. The analysis of influencing factors on geophysical field matching[C]//2024 International Conference on Guidance, Navigation and Control (ICGNC), Springer, 2024. (EI)}
		\vspace{-5pt}
	\item{Chen X, Jiao Z, \textbf{Zhao S}, Deng Z. Research on Polar Navigation Problem of North-Seeking Strapdown Inertial Navigation System[C]//2024 International Conference on Guidance, Navigation and Control (ICGNC), Springer, 2024. (EI)}
		\vspace{-5pt}
	\item{Wang Q, Gong X, Bai X, Deng Z, \textbf{Zhao S}. The method for selecting adaptation zones of terrain matching based on Arctic seabed terrain features[C]//2024 International Conference on Guidance, Navigation and Control (ICGNC), Springer, 2024. (EI)}
	
\resumeSubheadingone
	{Patents}{}
	\vspace{-5pt}
	
	{CN202110545814.3 (3rd), CN202311811811.5 (2nd), CN202311694509.6 (3rd), CN202110545816.2 (3rd), CN202110838109.2 (3rd), CN202211221726.9 (4th), CN202210856742.9 (2nd), CN201911385476.0 (4th).}
  \resumeSubHeadingListEnd
  %-------------------------------------------
  
 % %-----------ACADEMIC PROJECTS AND INTERNSHIPS----------------- 
\shorterSection{\textcolor{titleblue}{Awards}}
\resumeSubHeadingListStart
	\resumeSubheadingtwo
	{Technical Invention Award of Chinese Society of Inertial Technology (\textbf{First Prize})}{Aug, 2022}
	\vspace{-12pt}
 	\resumeSubheadingtwo
 {APMCM Mathematical Modeling (\textbf{First Prize, TOP 4})}{Jan, 2022}
 \vspace{-12pt}
 \resumeSubheadingtwo
 {Huawei Cup Mathematical Modeling (\textbf{Third Prize})}{Dec, 2021}
  \vspace{-12pt}
 \resumeSubheadingtwo
 {MathorCup Mathematical Modeling (\textbf{Third Prize})}{Jun, 2021}
   \vspace{-12pt}
 \resumeSubheadingtwo
 {Outstanding graduate of School of Xuteli}{Jun, 2020}
    \vspace{-12pt}
 \resumeSubheadingtwo
 {The Artificial Intelligence Challenge of Robomaster, perception (\textbf{Excellence Award})}{Jun, 2020}
     \vspace{-12pt}
 \resumeSubheadingtwo
 {NXP National College Student Smart Car Competition (\textbf{Second Prize in North China)}}{Jul, 2020}
      \vspace{-12pt}
 \resumeSubheadingtwo
 {China Undergraduate Mathematical Contest in Modeling (\textbf{Second Prize in Beijing)}}{Jan, 2019}
 
  \resumeSubHeadingListEnd
    %-------------------------------------------
  

% %-----------ACADEMIC PROJECTS AND INTERNSHIPS-----------------
% \shorterSection{Academic Projects and internships}
%   \resumeSubHeadingListStart
%   \small
%     \item{
%      \textbf{Languages}{: Python, C++, SQL, Java, Swift}
%      \hfill
%      \textbf{Technologies}{: GCP, AWS, GitHub, GitLab, Docker}
%     }
%     \vspace{-5pt}
%     \item{
%      \textbf{Libraries}{: TensorFlow, PyTorch, Keras, Scikit-Learn, Numpy, Pandas, Spark, Jupyter, OpenCV, PIL, OpenCL, OpenGL, CUDA}
%     }
% \resumeSubHeadingListEnd

%-----------EXPERIENCE-----------------
\shorterSection{\textcolor{titleblue}{Experience}}
  \resumeSubHeadingListStart

    \resumeSubheading
      {Underwater geophysical field matching navigation}{Beijing China}
      {PH.D student, Supervisor: \textbf{Zhihong Deng}}{May 2019 - Now}
      
      {The accumulated error of inertial navigation system is corrected by using underwater geophysical field for matching and positioning. My work content is to study the underwater carrier path planning method and matching positioning method. In the field of path planning, I consider the change rate of gravity field and mismatching phenomenon, and use Astar and DWA algorithms to plan the trajectory of underwater vehicles. In order to integrate inertial navigation information, geophysical field reference map and gravimeter measurement information, some matching algorithms are designed to realize positioning. It includes introducing the correlation filter in target recognition and particle filter, considering the potential relationship of geophysical field characteristics. 
      	
      In the past, I worked on gravity field aided navigation. Now I work on terrain aided navigation and positioning. The localization of underwater vehicle is studied by inertial navigation system, single-beam echosounder or multi-beam echosounder.
      }
  
      \resumeSubheading
  		{Robomaster robot positioning and decision making}{Beijing China}
 		{Undergraduate students, Supervisor: \textbf{Robomaster Team}}{Sep 2019 - May 2020}
 		
 		{In the specified area, four robots need to fight against enemy robots, by designing different decision-making strategies to defeat the enemy robots. I work on robot positioning and decision making. The calculation amount is reduced by optimizing the AMCL algorithm, and the comparison of several algorithms is made. Handle battlefield complexities by designing different decisions, such as how robots behave when they don't have bullets.}
 		
 		\resumeSubheading
 		{Hardware circuit design of intelligent energy-saving car}{Beijing China}
 		{Undergraduate students, Supervisor: \textbf{Smart Car Club}}{Sep 2019 - July 2020}
 		
 		{In the case that the battery is not applicable as a power source, the coil and the changing magnetic field are used to provide power for the smart car, so that the car can complete the driving within the specified time. My work is to design the circuit of the whole car and the design of the PCB, including the voltage regulator circuit design, control circuit design and signal circuit design. After that, as vice captain of the team club, participated in guiding the preparation and organization of the next race.}
  
%      \resumeItemListStart
%        \resumeItem{Deep Monocular VO}
%          { Designed a semi-supervised monocular depth estimator for video using sparse bundle adjustment in a sliding window, achieved 0.117 (top5) RMSE in \textbf{NYU v2} and 2.981 (top10) RMSE in \textbf{Kitti eigen-split}. backbone is Unet+CSPN, trained with semi-dense map point tracked by VO, implemented with \textbf{PyTorch}. }
%        \resumeItem{NavGuideNet}
%          {A \textbf{synthesized hierarchical neural network} for autonomous navigation in complex environment and variant landscapes (tested in field/underwater environments). Backbone \textbf{encoder} is \textbf{Resnet18}, \textbf{latent code} was concaternated with control signals and \textbf{decoder} is \textbf{de-convolution network} (transposed convolution)}
%        \resumeItem{Deep RL Auto Driving (Sim2Real)}
%          {Re-implemented CAD2RL in python and extended to multiple policy gradient based backends \textbf{(A3C+LSTM)} in continuous action space, simulator is \textbf{Microsoft AirSim}, tested on RC car and UAV.
%          }
%      \resumeItemListEnd
%
%    \resumeSubheading
%      {iLab Tongji/University of South California}{Shanghai, China, Los Angeles, USA}
%      {Research Assistant, Supervisor: \textbf{Jianwei Lu, Laurent Itti}}{Apr 2015 - Jul 2017}
%      \resumeItemListStart
%        \resumeItem{SLAM Fusion}
%          {Vision (monocular) LiDar fusion with direct method (jointly optimize optical flow with Sparse Bundle Adjustment on ORB features) extra constraint from LiDar helps eliminating depth from null space. }
%        \resumeItem{Visual SLAM with Saliency}
%          { joint optimizing the graph (G2O) with salient voting as extra binary edges.}
%      \resumeItemListEnd
%
%    \resumeSubheading
%      {UCLA}{Los Angeles, USA}
%      {Research Assistant, Supervisor: \textbf{Yi Xing}}{Jul 2015 - Jan 2016}
%      \resumeItemListStart
%        \resumeItem{Code Parallelization}
%          {optimized their RNA analysis tool, [stable release (\href{http://rnaseq-mats.sourceforge.net/rmats3.0.9/}{\textcolor{linkblue}{rMATS 3.0.9}})], binding the large matrix calculations with C11 (SSE/AVX vectorization, Intel) and CUDA (cuBLAS , Nvidia) }
%      \resumeItemListEnd

  \resumeSubHeadingListEnd

%%-----------PROJECTS/SKILLS-----------------
%\shorterSection{\textcolor{titleblue}{Projects}}
%  \resumeSubHeadingListStart
%    \resumeSubItem{Visual SLAM}
%     {
%     Comprehensively \textbf{re-implemented} \href{https://github.com/rancheng/dso_understands}{\textcolor{linkblue}{DSO}} and annotated with exhaustive explains.
%     }
%    \resumeSubItem{Deep Monocular Dense 3D Reconstruction}{Dense 3D reconstruction with \textbf{monodepth2} initialized Visual Odometry, leveraging traditional photometric consistancy, occlusion discrepancy, and local geometrical-smooth assumptions to \textbf{optimize depth estimation} (\textbf{LM} method) and \textbf{register} 3D map point clouds.}
%    \resumeSubItem{Abstraction Augmented Deep RL}
%      {Abstract rgb image with Unet shaped network to digest image in latent representation, and learn from latent inputs, average convergence time increased 27.3\%, maximum reward (10M iterations) is 1.21 times than baseline model without abstraction augmentation, experiments conducted under self-collected dataset from AirSim simulator (\href{https://github.com/rancheng/AirSimProjects}{\textcolor{linkblue}{github}})}
%    \resumeSubItem{Forgetting Model for BP}
%      {Introduced forgetting model for back propagation as in gradient dynamic routine, inspired from forgetting curve, I invented forgetting factor to regulate delta weights updates (\href{https://rancheng.github.io/forgetting-model/}{\textcolor{linkblue}{math proof}})}
%    \resumeSubItem{LOAM}{extended LOAM (LiDAR Odometry and Mapping) with co-visibility check, optimized with Ceres optimizer and asynchronous threading}
%  \resumeSubHeadingListEnd

\end{document}
